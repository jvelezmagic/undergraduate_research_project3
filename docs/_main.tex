%%%%%%%%%%%%%%%%%%%%%%%%%%%%%%%%%%%%%%%%%%%%%%%%%%%%%%%%%%%%%%%
%% OXFORD THESIS TEMPLATE

% Use this template to produce a standard thesis that meets the Oxford University requirements for DPhil submission
%
% Originally by Keith A. Gillow (gillow@maths.ox.ac.uk), 1997
% Modified by Sam Evans (sam@samuelevansresearch.org), 2007
% Modified by John McManigle (john@oxfordechoes.com), 2015
% Modified by Ulrik Lyngs (ulrik.lyngs@cs.ox.ac.uk), 2018-, for use with R Markdown
%
% Ulrik Lyngs, 25 Nov 2018: Following John McManigle, broad permissions are granted to use, modify, and distribute this software
% as specified in the MIT License included in this distribution's LICENSE file.
%
% John commented this file extensively, so read through to see how to use the various options.  Remember that in LaTeX,
% any line starting with a % is NOT executed.  Several places below, you have a choice of which line to use
% out of multiple options (eg draft vs final, for PDF vs for binding, etc.)  When you pick one, add a % to the beginning of
% the lines you don't want.


%%%%% PAGE LAYOUT
% The most common choices should be below.  You can also do other things, like replacing "a4paper" with "letterpaper", etc.

% This one formats for two-sided binding (ie left and right pages have mirror margins; blank pages inserted where needed):
%\documentclass[a4paper,twoside]{templates/ociamthesis}
% This one formats for one-sided binding (ie left margin > right margin; no extra blank pages):
%\documentclass[a4paper]{ociamthesis}
% This one formats for PDF output (ie equal margins, no extra blank pages):
%\documentclass[a4paper,nobind]{templates/ociamthesis}

% As you can see from the uncommented line below, oxforddown template uses the a4paper size, 
% and passes in the binding option from the YAML header in index.Rmd:
\documentclass[a4paper, nobind]{templates/ociamthesis}


%%%%% ADDING LATEX PACKAGES
% add hyperref package with options from YAML %
\usepackage[pdfpagelabels]{hyperref}
% change the default coloring of links to something sensible
\usepackage{xcolor}

\definecolor{myurlcolor}{RGB}{0,0,139}
\definecolor{mycitecolor}{RGB}{0,33,71}

\hypersetup{
  hidelinks,
  colorlinks,
  linkcolor=.,
  urlcolor=myurlcolor,
  citecolor=mycitecolor
}



% add float package to allow manual control of figure positioning %
\usepackage{float}

% enable strikethrough
\usepackage[normalem]{ulem}

% use soul package for correction highlighting
\usepackage{color, soul}
\definecolor{correctioncolor}{HTML}{CCCCFF}
\sethlcolor{correctioncolor}
\newcommand{\ctext}[3][RGB]{%
  \begingroup
  \definecolor{hlcolor}{#1}{#2}\sethlcolor{hlcolor}%
  \hl{#3}%
  \endgroup
}
\soulregister\ref7
\soulregister\cite7
\soulregister\autocite7
\soulregister\textcite7
\soulregister\pageref7

%%%%% FIXING / ADDING THINGS THAT'S SPECIAL TO R MARKDOWN'S USE OF LATEX TEMPLATES
% pandoc puts lists in 'tightlist' command when no space between bullet points in Rmd file,
% so we add this command to the template
\providecommand{\tightlist}{%
  \setlength{\itemsep}{0pt}\setlength{\parskip}{0pt}}
 
% UL 1 Dec 2018, fix to include code in shaded environments

% User-included things with header_includes or in_header will appear here
% kableExtra packages will appear here if you use library(kableExtra)
\usepackage{booktabs}
\usepackage{longtable}
\usepackage{array}
\usepackage{multirow}
\usepackage{wrapfig}
\usepackage{float}
\usepackage{colortbl}
\usepackage{pdflscape}
\usepackage{tabu}
\usepackage{threeparttable}
\usepackage{threeparttablex}
\usepackage[normalem]{ulem}
\usepackage{makecell}
\usepackage{xcolor}


%UL set section header spacing
\usepackage{titlesec}
% 
\titlespacing\subsubsection{0pt}{24pt plus 4pt minus 2pt}{0pt plus 2pt minus 2pt}


%UL set whitespace around verbatim environments
\usepackage{etoolbox}
\makeatletter
\preto{\@verbatim}{\topsep=0pt \partopsep=0pt }
\makeatother



%%%%%%% PAGE HEADERS AND FOOTERS %%%%%%%%%
\usepackage{fancyhdr}
\setlength{\headheight}{15pt}
\fancyhf{} % clear the header and footers
\pagestyle{fancy}
\renewcommand{\chaptermark}[1]{\markboth{\thechapter. #1}{\thechapter. #1}}
\renewcommand{\sectionmark}[1]{\markright{\thesection. #1}} 
\renewcommand{\headrulewidth}{0pt}

\fancyhead[LO]{\emph{\leftmark}} 
\fancyhead[RE]{\emph{\rightmark}} 

% UL page number position 
\fancyfoot[C]{\emph{\thepage}} %regular pages
\fancypagestyle{plain}{\fancyhf{}\fancyfoot[C]{\emph{\thepage}}} %chapter pages

% JEM fix header on cleared pages for openright
\def\cleardoublepage{\clearpage\if@twoside \ifodd\c@page\else
   \hbox{}
   \fancyfoot[C]{}
   \newpage
   \if@twocolumn\hbox{}\newpage
   \fi
   \fancyhead[LO]{\emph{\leftmark}} 
   \fancyhead[RE]{\emph{\rightmark}} 
   \fi\fi}


%%%%% SELECT YOUR DRAFT OPTIONS
% This adds a "DRAFT" footer to every normal page.  (The first page of each chapter is not a "normal" page.)

% IP feb 2021: option to include line numbers in PDF

% This highlights (in blue) corrections marked with (for words) \mccorrect{blah} or (for whole
% paragraphs) \begin{mccorrection} . . . \end{mccorrection}.  This can be useful for sending a PDF of
% your corrected thesis to your examiners for review.  Turn it off, and the blue disappears.
\correctionstrue


%%%%% BIBLIOGRAPHY SETUP
% Note that your bibliography will require some tweaking depending on your department, preferred format, etc.
% If you've not used LaTeX before, I recommend reading a little about biblatex/biber and getting started with it.
% If you're already a LaTeX pro and are used to natbib or something, modify as necessary.
% Either way, you'll have to choose and configure an appropriate bibliography format...


\usepackage[style=authoryear, sorting=nyt, backend=biber, maxcitenames=2, useprefix, doi=true, isbn=false, uniquename=false]{biblatex}
\newcommand*{\bibtitle}{Works Cited}

\addbibresource{bibliography/references.bib}
\addbibresource{bibliography/additional-references.bib}


% This makes the bibliography left-aligned (not 'justified') and slightly smaller font.
\renewcommand*{\bibfont}{\raggedright\small}


% Uncomment this if you want equation numbers per section (2.3.12), instead of per chapter (2.18):
%\numberwithin{equation}{subsection}


%%%%% THESIS / TITLE PAGE INFORMATION
% Everybody needs to complete the following:
\title{Bacterial filamentation: a bet for survival in stressful environments}
\author{Jesús Vélez Santiago}
\college{Center for Genomic Sciences}

% Master's candidates who require the alternate title page (with candidate number and word count)
% must also un-comment and complete the following three lines:

% Uncomment the following line if your degree also includes exams (eg most masters):
%\renewcommand{\submittedtext}{Submitted in partial completion of the}
% Your full degree name.  (But remember that DPhils aren't "in" anything.  They're just DPhils.)
\degree{Bachelor of Science}
% Term and year of submission, or date if your board requires (eg most masters)
\degreedate{Cuernavaca, Morelos, Mexico - 2021-09}


%%%%% YOUR OWN PERSONAL MACROS
% This is a good place to dump your own LaTeX macros as they come up.

% To make text superscripts shortcuts
	\renewcommand{\th}{\textsuperscript{th}} % ex: I won 4\th place
	\newcommand{\nd}{\textsuperscript{nd}}
	\renewcommand{\st}{\textsuperscript{st}}
	\newcommand{\rd}{\textsuperscript{rd}}

%%%%% THE ACTUAL DOCUMENT STARTS HERE
\begin{document}

%%%%% CHOOSE YOUR LINE SPACING HERE
% This is the official option.  Use it for your submission copy and library copy:
\setlength{\textbaselineskip}{22pt plus2pt}
% This is closer spacing (about 1.5-spaced) that you might prefer for your personal copies:
%\setlength{\textbaselineskip}{18pt plus2pt minus1pt}

% You can set the spacing here for the roman-numbered pages (acknowledgements, table of contents, etc.)
\setlength{\frontmatterbaselineskip}{17pt plus1pt minus1pt}

% UL: You can set the line and paragraph spacing here for the separate abstract page to be handed in to Examination schools
\setlength{\abstractseparatelineskip}{13pt plus1pt minus1pt}
\setlength{\abstractseparateparskip}{0pt plus 1pt}

% UL: You can set the general paragraph spacing here - I've set it to 2pt (was 0) so
% it's less claustrophobic
\setlength{\parskip}{2pt plus 1pt}

%
% Oxford University logo on title page
%
\def\crest{{\includegraphics[width=5cm]{figures/logos/unam.pdf}}}
\renewcommand{\university}{National Autonomous University of Mexico}
\renewcommand{\submittedtext}{A thesis submitted for the degree of}


% Leave this line alone; it gets things started for the real document.
\setlength{\baselineskip}{\textbaselineskip}


%%%%% CHOOSE YOUR SECTION NUMBERING DEPTH HERE
% You have two choices.  First, how far down are sections numbered?  (Below that, they're named but
% don't get numbers.)  Second, what level of section appears in the table of contents?  These don't have
% to match: you can have numbered sections that don't show up in the ToC, or unnumbered sections that
% do.  Throughout, 0 = chapter; 1 = section; 2 = subsection; 3 = subsubsection, 4 = paragraph...

% The level that gets a number:
\setcounter{secnumdepth}{2}
% The level that shows up in the ToC:
\setcounter{tocdepth}{1}


%%%%% ABSTRACT SEPARATE
% This is used to create the separate, one-page abstract that you are required to hand into the Exam
% Schools.  You can comment it out to generate a PDF for printing or whatnot.

% JEM: Pages are roman numbered from here, though page numbers are invisible until ToC.  This is in
% keeping with most typesetting conventions.
\begin{romanpages}

% Title page is created here
\maketitle

%%%%% DEDICATION -- If you'd like one, un-comment the following.
\begin{dedication}
  For Yihui Xie
\end{dedication}

%%%%% ACKNOWLEDGEMENTS -- Nothing to do here except comment out if you don't want it.
\begin{acknowledgements}
 	Lorem ipsum dolor sit amet, consectetur adipiscing elit. Integer tristique, sem egestas aliquam varius, arcu nisi ullamcorper lacus, quis convallis enim velit et arcu. Vestibulum lacus arcu, tempor non dapibus vitae, malesuada ut ipsum. Phasellus condimentum diam ex. Sed maximus a mauris vel aliquet.

  Integer neque sapien, cursus eu viverra consequat, cursus congue dui. Maecenas est dui, rutrum vitae enim vel, varius scelerisque tortor. In vel dignissim orci. Integer varius neque mauris, mollis commodo libero fringilla sed. Nunc accumsan, libero id interdum dignissim, nulla nibh consectetur lorem, vel dignissim erat magna vitae ante. Aliquam at est lectus. Suspendisse nec sem euismod, condimentum neque sit amet, malesuada nibh. Aenean condimentum pharetra quam, id venenatis mauris tempor a.

  \begin{flushright}
  Jesús Vélez Santiago \\
  Center for Genomic Sciences \\
  2021, 08
  \end{flushright}
\end{acknowledgements}


%%%%% ABSTRACT -- Nothing to do here except comment out if you don't want it.
\begin{abstract}
	Etiam venenatis purus eu felis viverra sodales. Sed et varius ex. Nullam sit amet aliquet purus. Fusce elementum vitae est eget vulputate. Aliquam erat volutpat. Donec ac suscipit leo, sed euismod nisi. Duis malesuada elementum pulvinar. Ut et vulputate augue. In dignissim ligula at nulla feugiat cursus. Suspendisse sed diam ut ligula sollicitudin pellentesque. Etiam ut gravida velit, vitae lobortis mi. Nam congue laoreet mauris sit amet iaculis.
\end{abstract}

%%%%% MINI TABLES
% This lays the groundwork for per-chapter, mini tables of contents.  Comment the following line
% (and remove \minitoc from the chapter files) if you don't want this.  Un-comment either of the
% next two lines if you want a per-chapter list of figures or tables.
  \dominitoc % include a mini table of contents

% This aligns the bottom of the text of each page.  It generally makes things look better.
\flushbottom

% This is where the whole-document ToC appears:
\tableofcontents

\listoffigures
	\mtcaddchapter
  	% \mtcaddchapter is needed when adding a non-chapter (but chapter-like) entity to avoid confusing minitoc

% Uncomment to generate a list of tables:
\listoftables
  \mtcaddchapter
%%%%% LIST OF ABBREVIATIONS
% This example includes a list of abbreviations.  Look at text/abbreviations.tex to see how that file is
% formatted.  The template can handle any kind of list though, so this might be a good place for a
% glossary, etc.
% First parameter can be changed eg to "Glossary" or something.
% Second parameter is the max length of bold terms.
\begin{mclistof}{List of Abbreviations}{3.2cm}

\item[1-D, 2-D]

One- or two-dimensional, referring \textbf{in this thesis} to spatial dimensions in an image.

\item[Otter]

One of the finest of water mammals.

\item[Hedgehog]

Quite a nice prickly friend.

\end{mclistof} 


% The Roman pages, like the Roman Empire, must come to its inevitable close.
\end{romanpages}

%%%%% CHAPTERS
% Add or remove any chapters you'd like here, by file name (excluding '.tex'):
\flushbottom

% all your chapters and appendices will appear here
\hypertarget{introduction}{%
\chapter*{Introduction}\label{introduction}}
\addcontentsline{toc}{chapter}{Introduction}

\adjustmtc
\markboth{Introduction}{}

\hypertarget{experiment-analysis}{%
\chapter{Experiment analysis}\label{experiment-analysis}}

\minitoc 

\hypertarget{introduction-1}{%
\section{Introduction}\label{introduction-1}}

\hypertarget{general-preprocessing-of-data}{%
\section{General preprocessing of data}\label{general-preprocessing-of-data}}

The raw data processing consisted mainly of creating two levels of observation for the cells of both chromosomal strains and multicopy plasmids.
The first level is at a cell granularity, that is, point properties.
The second level consists of the cells over time, thus observing properties at the population level.
We did this because it would allow us to understand what factors are affecting filamentation and why.

We decided to normalize the fluorescence values of DS-Red and GFP for both experiments based on the values observed before the beginning of the exposure to antibiotics.
It allowed us to have a basis to work with and compare the expressions between cells.
In the case of DS-Red, we also applied a logarithmic transformation to observe subtle changes that would allow us to dig deeper.

Ultimately, we decided to classify cells into four fundamental groups based on whether the cell was filamented and survived (see \ref{fig:03-cells-distribution-across-experiments}).
We define a \emph{filamented cell} as a cell with more than two standard deviations from the mean concerning the lengths observed before antibiotic entry into the system {[}{]}.
On the other hand, although there are multiple ways to define death in a cell context {[}{]}, we decided to consider a \emph{cell dead or missing} when we stopped having information about it, either because of its fluorescence levels or because it left the field of experimental tracking.
Therefore, a \emph{surviving cell} is one that existed before and after exposure to the antibiotic.





\begin{figure}[H]
\includegraphics[width=1\linewidth]{downloadFigs4latex__main/03-cells-distribution-across-experiments} \caption[Cell classification and its distribution across experiments.]{\textbf{Cell classification and its distribution across experiments.} We define a \emph{filamented cell} as a cell whose length exceeded two standard deviations from the mean at any time during the experiment. A \emph{surviving cell} is a cell that existed before and after exposure to the antibiotic. We removed from the analysis those cells that died before or were born after the exposure of the experiment. Therefore, we delimited the effect caused by the exposure to the antibiotic.}\label{fig:03-cells-distribution-across-experiments}
\end{figure}

\hypertarget{results}{%
\section{Results}\label{results}}

\hypertarget{cell-length-and-the-amount-of-gfp-are-crucial-in-determining-cell-survival}{%
\subsection{Cell length and the amount of GFP are crucial in determining cell survival}\label{cell-length-and-the-amount-of-gfp-are-crucial-in-determining-cell-survival}}

From the complete set of tracking data, we decided to sum the tracking of each cell at the level of its tracking identifier.
The points to be evaluated represented the DS-Red, GFP, and length values at different time instants: initial, filamentation, and end.
This preprocessing allowed us to observe and quantify each cell at critical times in the experiment and eliminate noise or signals outside the scope of this investigation.

We defined the \emph{initial time} as the first time we observed the cell in the experiment.
\emph{Filamentation time} equals when a cell reaches the filamentation threshold (see Figure \ref{fig:03-temporal-metric-distribution-length-1}) for the first time.
We defined the \emph{end time} as the last observation of the cell.
We decided to bound the end time for surviving cells to one frame (10 min) after the end of antibiotic exposure so that the observed signal would reflect the final stress responses.

When we compared the distributions of the study variables (\emph{i.e.}, DS-Red, GFP, and length) for both experiments, we observed small subtleties of filamentation in its role for cell survival.
In Figure \ref{fig:03-temporal-metric-distribution-ds-red-1}, we show that indistinctly and as expected, surviving cells managed to eliminate the antibiotic by the end time, while dead cells presented higher levels of antibiotics.





\begin{figure}[H]
\includegraphics[width=1\linewidth]{downloadFigs4latex__main/03-temporal-metric-distribution-ds-red-1} \caption[DS-Red temporal distribution.]{\textbf{DS-Red temporal distribution.} To evaluate the incident effect of the antibiotic marked by DS-red on cells by class, we show its values at three key moments: start, filamentation (SOS), and end. The upper asterisks represent the significance value when comparing a group X to the filamented and surviving cell reference. Asterisks in a line indicate whether or not there is a significant difference in the survival of non-filamented cells. The black dots represent the mean of each group, and the lines that join them are a comparative guide. The extent of the black bars represents the distribution of the data. Although, at the initial time, we observe multiple significant differences, this is likely due to the intrinsic noise of the system since, as expected, the values are close to zero. We observed a difference between the surviving and non-filamented cells for the chromosomal strain for the SOS time, but the same did not occur for the plasmid strain. The final amount of DS-red makes a clear difference between survival and death.}\label{fig:03-temporal-metric-distribution-ds-red-1}
\end{figure}

On the other hand, the GFP observations in Figure \ref{fig:03-temporal-metric-distribution-gfp-1} showed us two essential things for cell sorting.
1) The chromosomal strain did not exhibit noticeable changes in GFP levels as expected, and 2) that the filamented cells were those that had the lowest amount of plasmids at the beginning of the experiment.
On the other hand, for the final observation times, GFP measurements indicated that among the cells that did not filament, the surviving ones reduced their GFP expression slightly more concerning the dead ones and the initial times.
Meanwhile, for the filamented cells, whether surviving or dead, their GFP measurements indicated no difference at the beginning or the end of the experiment, indicating the presence of other determinants of cell survival.





\begin{figure}[H]
\includegraphics[width=1\linewidth]{downloadFigs4latex__main/03-temporal-metric-distribution-gfp-1} \caption[GFP temporal distribution.]{\textbf{GFP temporal distribution.} To evaluate the incident effect of the GFP on cells by class, we used the same notation as in Figure \ref{fig:03-temporal-metric-distribution-ds-red-1}. The chromosomal strain has significant values of GFP at different times. However, these differences can be mainly due to experimental noise. We appreciated in the plasmid strain that the filamented cells had a lower amount of initial GFP as expected. At the time of filamentation, there is no difference between the survivors and the dead. However, in the end time, we observed that the surviving non-filamented cells have less GFP than the dead non-filamented and the living filamented cells.}\label{fig:03-temporal-metric-distribution-gfp-1}
\end{figure}

Cell length was one of the factors that GFP expression levels could not explain for cell survival.
In Figure \ref{fig:03-temporal-metric-distribution-length-1}, we show that the conclusions regarding filamentation were applicable for both chromosomal or plasmid strains.
For the initial times, filamented and survived cells were shorter in length than those that died but longer than not filamented cells of both classes, while non-filamented cells did not differ from each other.
We observed no length differences between cells at filamentation time.
Thus, survival could depend on other factors such as growth rate.
At the final time, the results were precise.
Surviving cells had a greater length relative to their non-surviving pair (\emph{i.e.}, dead filamented and non-filamented cells).
However, for filamented cells, surviving cells represent a distribution of higher final length values in general but not as extensive as their dead counterpart.
Which we could explain as a length limit to which cells can grow without dying.
Nevertheless, we had no information to evaluate such a hypothesis.





\begin{figure}[H]
\includegraphics[width=1\linewidth]{downloadFigs4latex__main/03-temporal-metric-distribution-length-1} \caption[Length temporal distribution.]{\textbf{Length temporal distribution.} To evaluate the incident effect of length on cells by class, we use the same notation as in Figure \ref{fig:03-temporal-metric-distribution-ds-red-1}. The observations for both strains, chromosomal or plasmid, are the same. In the beginning, the surviving filamented cells already have a difference in length concerning the rest of the classes. At the time of filamentation, there is no difference to help determine whether the cell will survive or not. Finally, in the final time, it seems that the surviving filamented cells have a greater length than the rest of the groups. However, this length is moderate compared to the excess length shown by non-surviving filamented cells. On the other hand, we highlighted the growth of the surviving non-filamented cells. Therefore, although they did not reach a length for us to classify as filamented, the cells did resort to filamentation.}\label{fig:03-temporal-metric-distribution-length-1}
\end{figure}

Once we observed the effects of GFP expression levels and lengths in determining whether a cell lives or dies, we decided to project the cells onto the plane and paint them with their class color to determine whether these two variables contained the necessary information to cluster the data correctly.
In Figure \ref{fig:03-focus-just-in-initial-values-1}, we show the initial GFP and length values projection.
While, with some work, we could contextually place the results in Figures \ref{fig:03-temporal-metric-distribution-gfp-1} and \ref{fig:03-temporal-metric-distribution-length-1}, the initial values did not appear to determine the classes.
Therefore, we explored the final versus initial values differences in Figure \ref{fig:03-metric-differences-1}.
With this new representation of the cells in the plane, we contextualized the statistical results presented in Figures \ref{fig:03-temporal-metric-distribution-gfp-1} and \ref{fig:03-temporal-metric-distribution-length-1}.
Besides, it showed us that differences in length (\emph{i.e.}, filamentation) and reductions in GFP expression are essential in determining cell survival.
Though, cell sorting is not entirely separate, which means that other variables are affecting the experimental results in cell survival.





\begin{figure}[H]
\includegraphics[width=1\linewidth]{downloadFigs4latex__main/03-focus-just-in-initial-values-1} \caption[Experiment initial values.]{\textbf{Experiment initial values.} By positioning a cell in space based on its initial length and GFP values, we can see that class separation occurs, but not as a strong signal. Therefore, we concluded that although the initial state influences the result, this is not everything. For this, we have the example of the length changes throughout the experiment caused by filamentation. In this graph, the GFP scale is at log10 to help us observe those minor differences between the experiments.}\label{fig:03-focus-just-in-initial-values-1}
\end{figure}





\begin{figure}[H]
\includegraphics[width=1\linewidth]{downloadFigs4latex__main/03-metric-differences-1} \caption[Experiment initial values differences.]{\textbf{Experiment initial values differences.} By comparing the metric differences of the last observation and the first observation of a cell, we can separate mainly the surviving filamented cells from those that did not do it in both experiments (green dots). Meanwhile, cells with plasmids form a small accumulation of surviving cells that did not produce filament (purple dots). However, this has made a breakthrough in understanding what is affecting cell survival. There are still variables that we can include to understand this phenomenon better.}\label{fig:03-metric-differences-1}
\end{figure}

\hypertarget{number-of-divisions-and-cell-age-do-not-appear-to-play-a-clear-role-in-determining-cell-survival}{%
\subsection{Number of divisions and cell age do not appear to play a clear role in determining cell survival}\label{number-of-divisions-and-cell-age-do-not-appear-to-play-a-clear-role-in-determining-cell-survival}}

In the \ref{cell-length-and-the-amount-of-gfp-are-crucial-in-determining-cell-survival} section, we explored how GFP variability and length help in determining cell survival.
However, Figures \ref{fig:03-focus-just-in-initial-values-1} and \ref{fig:03-metric-differences-1} showed us the possibility of other factors relevant to the phenomenon under study.
As some papers in the literature suggest, some of these other factors may be cell division and cell youth (\emph{i.e.}, how much time has passed since the last cell division at the time of exposure to a toxic agent) {[}x, x, x{]}.
Therefore, we chose to observe these two metrics in experiments at a purely qualitative level, i.e., without the inclusion of, e.g., metrics of membrane or cell cycle properties {[}x, x{]}.

Although we expected to see a small contribution, either by the number of divisions or cell youth, in Figures \ref{fig:03-division-number-divisions-1} and \ref{fig:03-division-time-since-last-division-1}, we could not observe a precise effect of these variables on cell survival.
Patterns that, although they could have an explanation or biological significance, we decided to omit as relevant in the characterization of our cells.
However, we derived from this analysis a slightly simpler variable that tells us whether a cell underwent a cell division event or not.
So it gives us a more generalized picture of the contribution of division to cell survival (see Figure \ref{fig:03-see-datasets-individually-pca-plasmid-variable-contribution-1}).





\begin{figure}[H]
\includegraphics[width=1\linewidth]{downloadFigs4latex__main/03-division-number-divisions-1} \caption[Cell's number of divisions.]{\textbf{Cell's number of divisions.} Chromosomal cells exhibited more divisions for surviving classes and non-surviving filamented cells (\emph{i.e.}, purple, green, and red dots) relative to unchanged behavior in plasmid cells. Therefore, its contribution to filamentation remains uncertain.}\label{fig:03-division-number-divisions-1}
\end{figure}





\begin{figure}[H]
\includegraphics[width=1\linewidth]{downloadFigs4latex__main/03-division-time-since-last-division-1} \caption[Time elapsed since the last division at the beginning of the experiment.]{\textbf{Time elapsed since the last division at the beginning of the experiment.} The mean time of the last division before starting the experiment indicates that it did not influence the final result for chromosomal cells. There is a slight difference between the filamented-not survived cells and the rest for cells with plasmids. However, the signal does not appear to be strong on the survival role. Therefore, we conclude that we have no evidence to support that the time of the last division at the beginning of the experiment influences the final classification results.}\label{fig:03-division-time-since-last-division-1}
\end{figure}

\hypertarget{time-to-reach-filamentation-matters-in-determining-cell-survival}{%
\subsection{Time to reach filamentation matters in determining cell survival}\label{time-to-reach-filamentation-matters-in-determining-cell-survival}}

In Figures \ref{fig:03-temporal-metric-distribution-ds-red-1}, \ref{fig:03-temporal-metric-distribution-gfp-1}, and \ref{fig:03-temporal-metric-distribution-length-1}, we showed how, at the time of filamentation, DS-Red and GFP levels appeared indifferent to the cells.
Therefore, we hypothesized that a possible variable that could determine cell survival could be its time to activate its anti-stress response system that causes filamentation.
Furthermore, we also guided our hypothesis by previous reports showing us how the gene expression level can induce filamentation with tight temporal coordination {[}x{]}.

While, for our analyses, we did not measure the concentration of antibiotic that triggers filamentation per se, we indirectly quantified its effect by using the time it took for a cell to reach a length at which it is already considered a filamentating cell.
Furthermore, to recognize that the observed effect was a product of the experiment, we decided to keep only filamented cells just once antibiotic exposure began.

Figure \ref{fig:03-time-to-filamentation-filamented-once-experiment-start-1} shows how filamentation times are much narrower for chromosome cells than for plasmid cells.
Then, we hypothesize that the effect could come from the heterogeneity in the plasmid copy number in the population.
Also, interestingly, we observed that, for both experiments, the cells that survived had longer filamentation times than the cells that died.
These differences in response times could tell us about two main things: 1) if the cell grows too fast, it will reach a limit and start to accumulate antibiotics constantly, and 2) if the cell grows too fast, it is likely that the cost of maintaining an ample length for prolonged periods of exposure will become counterproductive.





\begin{figure}[H]
\includegraphics[width=1\linewidth]{downloadFigs4latex__main/03-time-to-filamentation-filamented-once-experiment-start-1} \caption[Time to filamentation filtered.]{\textbf{Time to filamentation filtered.} To quantify the effect of filamented to survive, we filtered those cells that filamented during the experiment. In this way, we normalize the start times for the calculation of the filamentation time. For both strains, the filamentation time had a more significant delay in the surviving cells.}\label{fig:03-time-to-filamentation-filamented-once-experiment-start-1}
\end{figure}

On the other hand, in Figure \ref{fig:03-time-to-filamentation-include-time-to-filamentation-in-initial-exploration-initial-values-1}, we decided to project the results of Figure \ref{fig:03-time-to-filamentation-filamented-once-experiment-start-1} in a space similar to the one described in Figure \ref{fig:03-focus-just-in-initial-values-1}.
Thus, we separated our data into those cells that survived and those that did not and painted them when it took them to reach their filamented state.
With that fact, we realized that, indeed, by adding this temporal component to the initial variables of length and GFP, we could separate surviving cells from dead cells to a greater degree.
However, it may still not be enough, and there are still many other variables that play a crucial role in understanding the ecology of stress and how some cells will be survivors or not.





\begin{figure}[H]
\includegraphics[width=1\linewidth]{downloadFigs4latex__main/03-time-to-filamentation-include-time-to-filamentation-in-initial-exploration-initial-values-1} \caption[Experiment initial values with time to filamentation.]{\textbf{Experiment initial values with time to filamentation.} As in Figure \ref{fig:03-focus-just-in-initial-values-1}, including the time it will take for cells to filament allows us to understand the phenomenon of survival better. Cells that filamented and survived generally have a much higher delay than their non-filamented peers for both strains (see Figure \ref{fig:03-time-to-filamentation-filamented-once-experiment-start-1}).}\label{fig:03-time-to-filamentation-include-time-to-filamentation-in-initial-exploration-initial-values-1}
\end{figure}

\hypertarget{increasing-the-complexity-of-the-system-and-analyzing-it-in-an-unsupervised-way-allows-a-correct-classification-of-cell-states}{%
\subsection{Increasing the complexity of the system and analyzing it in an unsupervised way allows a correct classification of cell states}\label{increasing-the-complexity-of-the-system-and-analyzing-it-in-an-unsupervised-way-allows-a-correct-classification-of-cell-states}}

In the experiments, we observed the importance of GFP filamentation and variability for cell survival.
Similarly, we realized that other variables must be affecting the final results.
Filamentation and GFP variability alone did not fully recapitulate the expected behavior of the data.
That is, the target variables did not capture the heterogeneity of the system.

The inability to reproduce cell sorting led us to question two things: 1) the possibility that our sorting was wrong beforehand and 2) we did not have enough variables to capture the study phenomenon.
We decided to take the unsupervised learning way to answer these subjects because it allows us to project our data without a priori knowledge {[}x{]}.

We opted for the path of dimensionality reduction techniques where each variable or feature is equivalent to one dimension.
The essence of dimensionality reduction is that it is not feasible to analyze each dimension with many dimensions.
Furthermore, dimensionality reduction helps us counteract several problems such as reducing the complexity of a model, reducing the possibility of overfitting a model, removing all correlated variables, and, of course, visualizing our data in a two- or three-dimensional space for better appreciation {[}x{]}.
Improved visualization and identification of essential variables are the main reasons to guide and complement our research with this technique.

\hypertarget{principal-component-analysis-pca-emphasizes-the-importance-of-cell-length-and-its-gfp-in-cell-survival}{%
\subsubsection{Principal Component Analysis (PCA) emphasizes the importance of cell length and its GFP in cell survival}\label{principal-component-analysis-pca-emphasizes-the-importance-of-cell-length-and-its-gfp-in-cell-survival}}

The first dimensionality reduction technique we decided to use was Principal Component Analysis (PCA) {[}x{]}.
Scientist mainly uses PCA to create predictive models or in Exploratory Data Analysis (EDA).
In our case, we only use it as an EDA.

For chromosomal and plasmid strain, in Figures \ref{fig:03-see-datasets-individually-pca-chromosome-points-in-new-coordinates-1} and \ref{fig:03-see-datasets-individually-pca-plasmid-points-in-new-coordinates-1}, we show the projection of the first two principal components (PCs), respectively.
Figure \ref{fig:03-see-datasets-individually-pca-chromosome-points-in-new-coordinates-1} separates the manually annotated classes, surviving cells separated from non-surviving cells.
However, for Figure \ref{fig:03-see-datasets-individually-pca-plasmid-points-in-new-coordinates-1}, the class separation was a bit rougher but allowed us to separate the surviving filament cells from the dead ones.





\begin{figure}[H]
\includegraphics[width=1\linewidth]{downloadFigs4latex__main/03-see-datasets-individually-pca-chromosome-points-in-new-coordinates-1} \caption[Principal Component Analysis of chromosomal strain.]{\textbf{Principal Component Analysis of chromosomal strain.} When integrating the information of different variables in a dimensionality reduction analysis, we observed a clear separation between the surviving cells and those that did not. The contributions that determined this phenomenon come mainly from the last amount of DS-red, GFP, and cell length (see Figure \ref{fig:03-see-datasets-individually-pca-chromosome-variable-contribution-1}). Although it seems obvious, it effectively confirms that the temporal classification that we carry out makes sense. Thus, highlighting the importance of filamentation. Longer length represents a greater uptake of antibiotics, but in a much larger volume, so the net effect is an internal reduction of antibiotics (see Figure \ref{fig:cell-dimensions-relationship}).}\label{fig:03-see-datasets-individually-pca-chromosome-points-in-new-coordinates-1}
\end{figure}





\begin{figure}[H]
\includegraphics[width=1\linewidth]{downloadFigs4latex__main/03-see-datasets-individually-pca-chromosome-variable-contribution-1} \caption[Variables contribution of Principal Component Analysis of chromosomal strain.]{\textbf{Variables contribution of Principal Component Analysis of chromosomal strain.} In the figure \ref{fig:03-see-datasets-individually-pca-chromosome-points-in-new-coordinates-1}, we see that the classes we created manually reflected what we observed when performing a reduction of dimensions analysis. Here we show the individual contribution of each variable for the first two components. The variables that most affected components 1 and 2 (X-axis and Y-axis, respectively) are the final measurements of DS-red, GFP, length, and the initial amount of GFP. Given that they are chromosomal strains, we should note that this variability could be produced by intrinsic experimental noise that we could not remove. With that in mind, having the DS-red and the final length highlights the inherent role of cells by having increased its size.}\label{fig:03-see-datasets-individually-pca-chromosome-variable-contribution-1}
\end{figure}

For their part, in Figures \ref{fig:03-see-datasets-individually-pca-chromosome-variable-contribution-1} and \ref{fig:03-see-datasets-individually-pca-plasmid-variable-contribution-1}, we show the total contribution of each variable per PC for the chromosomal and plasmid strain, respectively.
Finding that, indeed, filamentation plays a crucial role in determining cell survival.
For example, for PC2, we appreciated how the variable end DS-Red directed the dots to the positive side, while the variable end and start length directed the dots to the opposing side.
Therefore, we can support that filamentation has a role in moving cells away from having higher amounts of DS-Red.





\begin{figure}[H]
\includegraphics[width=1\linewidth]{downloadFigs4latex__main/03-see-datasets-individually-pca-plasmid-points-in-new-coordinates-1} \caption[Principal Component Analysis of plasmid strain.]{\textbf{Principal Component Analysis of plasmid strain.} By integrating the information from different variables in a dimensionality reduction analysis, we observed a clear separation between the filamented and non-filamented cells. Said class separation is given by component 2 (Y-axis), which is determined primarily by the initial and final lengths of the cells (see Figure \ref{fig:03-see-datasets-individually-pca-plasmid-variable-contribution-1}). Furthermore, the classification also allows us to separate those filamented cells that died from those that survived. Therefore, despite the increase in the system's complexity, length plays a role in determining survival.}\label{fig:03-see-datasets-individually-pca-plasmid-points-in-new-coordinates-1}
\end{figure}





\begin{figure}[H]
\includegraphics[width=1\linewidth]{downloadFigs4latex__main/03-see-datasets-individually-pca-plasmid-variable-contribution-1} \caption[Variables contribution of Principal Component Analysis of plasmid strain.]{\textbf{Variables contribution of Principal Component Analysis of plasmid strain.} In Figure \ref{fig:03-see-datasets-individually-pca-plasmid-points-in-new-coordinates-1}, we saw that we could separate the filamented cells from the non-filamented ones. The reduction analysis also shows a slight difference between surviving and dead cells within the small group of filamented cells. Here we offer the individual contribution of each variable for the first two components. For the first component (x-axis in Figure \ref{fig:03-see-datasets-individually-pca-chromosome-points-in-new-coordinates-1}), the initial and final GFP measurements received mainly the variability. We expected this component's importance since, being a chromosomal strain, we hope that its inherent variation will be inherited into the system. On the other hand, the second component (Y-axis in Figure \ref{fig:03-see-datasets-individually-pca-chromosome-points-in-new-coordinates-1}) was determined by the length of the cell. Factors that, in the chromosomal strain (see Figure \ref{fig:03-see-datasets-individually-pca-chromosome-variable-contribution-1}), determined with the help of DS-red the separation between surviving and dead cells.}\label{fig:03-see-datasets-individually-pca-plasmid-variable-contribution-1}
\end{figure}

\hypertarget{uniform-manifold-approximation-and-projection-umap-correctly-represents-the-local-structure-of-cell-states}{%
\subsubsection{Uniform Manifold Approximation and Projection (UMAP) correctly represents the local structure of cell states}\label{uniform-manifold-approximation-and-projection-umap-correctly-represents-the-local-structure-of-cell-states}}

Staying with only a one-dimensionality reduction technique was not an option, so we used the UMAP technique {[}x{]}.
We mainly decided to use UMAP for clustering purposes and see if the annotated clusters corresponded to the manually annotated ones.
UMAP has certain advantages for these purposes, e.g., it preserves the global structure across the whole space, so the distances between clusters matter.

In Figures \ref{fig:03-see-datasets-individually-umap-chromosome-points-in-new-coordinates-1} and \ref{fig:3-see-datasets-individually-umap-plasmid-points-in-new-coordinates-1}, we show how, using the same variables used in the ``PCA'' section, UMAP was able to cluster the four proposed classes correctly.
Interestingly, in Figure \ref{fig:03-see-datasets-individually-umap-chromosome-points-in-new-coordinates-1}, UMAP formed three general groups and four for Figure \ref{fig:3-see-datasets-individually-umap-plasmid-points-in-new-coordinates-1}.
However, in general, UMAP clustered the surviving cells from those that did not make it.
On investigating why this separation occurred, we found that the large groups coalesced into one another if we eliminated the division variable.
So, in a way, the division also has a role in determining survival, but it is not essential or at least not over-represented in our data.





\begin{figure}[H]
\includegraphics[width=1\linewidth]{downloadFigs4latex__main/03-see-datasets-individually-umap-chromosome-points-in-new-coordinates-1} \caption[UMAP coordinates of chromosome strain.]{\textbf{UMAP coordinates of chromosome strain.} We represented the cells in a low dimensional space. This new projection made it possible to group the cells that survived and those that did not. Therefore, as in PCA Figure \ref{fig:03-see-datasets-individually-pca-chromosome-points-in-new-coordinates-1}, this technique supports the manual classification that we carry out.}\label{fig:03-see-datasets-individually-umap-chromosome-points-in-new-coordinates-1}
\end{figure}





\begin{figure}[H]
\includegraphics[width=1\linewidth]{downloadFigs4latex__main/3-see-datasets-individually-umap-plasmid-points-in-new-coordinates-1} \caption[UMAP coordinates of plasmid strain.]{\textbf{UMAP coordinates of plasmid strain.} As in its \ref{fig:03-see-datasets-individually-umap-chromosome-points-in-new-coordinates-1} pair, the representation in a low-dimensional space helped classify the cells, grouping mainly into 4 groups, 2 of survivors and 2 of non-survivors. The variable \emph{division} marks the separation of classes. The \emph{division} variable indicates whether a cell is divided during its lifetime or not. Together, the UMAP represents the manually assigned classes.}\label{fig:3-see-datasets-individually-umap-plasmid-points-in-new-coordinates-1}
\end{figure}

\hypertarget{population-dynamics-reveal-how-filamentation-helps-cell-survival}{%
\subsection{Population dynamics reveal how filamentation helps cell survival}\label{population-dynamics-reveal-how-filamentation-helps-cell-survival}}

From the full tracking dataset, we decided to show how the different cell states behaved over time---for example, understanding how the cells concentrated antibiotics inside or how they increased in length for each instant.
In contrast to the dataset generated in the \ref{cell-length-and-the-amount-of-gfp-are-crucial-in-determining-cell-survival} section, we did not truncate the results 10 minutes after the antibiotic exposure.
In this way, we were able to observe cell behavior before and after the presence of the toxic agent.

In Figure \ref{fig:04-cell-status-over-time-status-with-dead-1}, we observed a small fraction of cells that were already filamentous without exposure to the toxic agent in both cell strains.
However, after the onset of antibiotic exposure at minute 60, we observed increases in the proportion of filamented cells.
It is interesting to note how filament cells grew after antibiotic exposure for the chromosomal strain.
We believe that this post-antibiotic growth exists because, once the SOS system that triggers filamentation is activated, the system continues to grow until it reaches a limit regardless of whether the damaging agent is still present or not.
Moreover, we observed how the cells start to divide again after some time because the proportion of non-filament cells starts to grow while the filament cells start to divide.
We observed the same effects for the plasmid strain.
However, by experimental design, the number of filament cells expected was much lower.





\begin{figure}[H]
\includegraphics[width=1\linewidth]{downloadFigs4latex__main/04-cell-status-over-time-status-with-dead-1} \caption[Population status over time.]{\textbf{Population status over time.} We calculate how many cells of each type existed for each instant of time: non-filamented and filamented living cells (blue and green areas, respectively) and dead cells (red area). We considered \emph{dead} cells as those cells that existed at one time and then stopped tracking. The black vertical lines represent the start and end of antibiotic exposure for each experiment. The effect of filamentation and its spread after exposure to the antibiotic is evident for the chromosomal strain. The experiment was finalized with the resolution of the cells when they returned to their non-filamented state. For its part, for the plasmid strain, it is observed how the filamented cells begin to appear slowly. Their proportion is as expected, given that the population had a wide distribution of GFP that allowed them to combat exposure to the antibiotic.}\label{fig:04-cell-status-over-time-status-with-dead-1}
\end{figure}

In Figure \ref{fig:04-metrics-over-time-1}, we showed how once antibiotics exposure began, those cells that died had a much faster increase in DS-Red than those that did manage to live, regardless of whether they filamented or not.
On the other hand, the surviving cells managed to maintain their DS-Red levels relatively stable.
We noted that length was critical for the surviving cells for the chromosomal strain by turning to the GFP and length variables for a temporal explanation.
Even cells categorized as non-filamented reached the filamentation threshold minutes after antibiotic exposure.
However, the distinction of live or dead filamented cells was not as evident as expected.
As for cells with plasmids, the effect on GFP for surviving cells was maintained for filamented cells and decreased for non-filamented cells.
For the filament cells that died, we showed that they had, on average, a much longer initial length than the surviving cells, so we also consider it as a necessary factor in understanding which variables affect cell survival.





\begin{figure}[H]
\includegraphics[width=1\linewidth]{downloadFigs4latex__main/04-metrics-over-time-1} \caption[Population measurements over time.]{\textbf{Population measurements over time.} The colored lines symbolize the average value of each metric at each instant of time, while its surrounded gray shaded area represents the 95\% confidence interval. The vertical lines represent the start and end of antibiotic exposure. The horizontal line in the length metric symbolizes the threshold to consider a cell filament. We observed a faster increase of DS-red for the non-surviving populations in both experiments. Regarding the GFP metric, the behavior is relatively stable for the chromosomal strain. In contrast, for the plasmid strain, a decline in GFP is observed for the population that did not survive. For the length metric, it is interesting to note how the chromosome cells that did not filament continued to grow past the filamentation threshold once the exposure to the antibiotic in the chromosomal strain had ended. On the other hand, the filamented and dead cells seem to have a greater length from the beginning for the plasmid strain.}\label{fig:04-metrics-over-time-1}
\end{figure}

\hypertarget{heterogeneity-of-cell-gfp-allows-various-forms-of-survival-and-addition-to-filamentation}{%
\subsection{Heterogeneity of cell GFP allows various forms of survival and addition to filamentation}\label{heterogeneity-of-cell-gfp-allows-various-forms-of-survival-and-addition-to-filamentation}}

We are confident that filamentation has a fundamental role in determining cell survival, with what we have shown so far.
However, for plasmid cells, we have a component that is of our complete interest; heterogeneity.
Each cell can possess a different plasmids number; thus, each could show a different behavior under stress {[}x, x{]}.
For example, heterogeneity can produce resistant cells that do not suffer damage, susceptible cells, and cells that form filaments to mitigate environmental stress.

To understand the variability in plasmid copy number in the population, we decided to group cells by the proportion of initial GFP regarding the population maximum.
Furthermore, we defined 100\% of the population as the number of total cells at the onset of antibiotic exposure.
Figure \ref{fig:04-cell-status-over-time-proportion-living-cells-by-gfp-row-1} shows how the cells with the highest amount of GFP remained unchanged once antibiotic exposure began, while the rest of the cells started to decrease their percentage of surviving cells.
However, the decrease was not linear.
On the contrary, we observed a normal distribution in terms of the reduction of live cells.
An average GFP point provided higher survival than a point below or above the average (except for cells very close to the population maximum).





\begin{figure}[H]
\includegraphics[width=1\linewidth]{downloadFigs4latex__main/04-cell-status-over-time-proportion-living-cells-by-gfp-row-1} \caption[Population survivals binned by initial GFP over time.]{\textbf{Population survivals binned by initial GFP over time.} We categorized the cells' GFP into ranges of proportions 0.05 concerning the maximum amount of GFP in the population. 100\% cells per bin of GFP was taken as the number of cells one frame before the start of exposure to the antibiotic (minute 50). Therefore, dark to light colors represent a generation of new cells, and light to dark colors the death of cells. The black vertical bars represent the start and end of the antibiotic exposure. Bars size and color on the right represent the percentage of the living cells 10 minutes after the end of the experiment. As shown in Figure \ref{fig:04-gfp-survival-probability-1}, we showed that the surviving cells appear to follow something similar to a bimodal distribution. More cells survive with a moderate amount of GFP or with an amount close to the maximum of the population.}\label{fig:04-cell-status-over-time-proportion-living-cells-by-gfp-row-1}
\end{figure}

Therefore, what we observed was a bimodal distribution for GFP-dependent cell survival.
In order to show this effect more clearly, in Figure \ref{fig:04-gfp-survival-probability-1}, we plotted the survival probability for each GFP bin without normalizing for the population maximum.
This new plot allowed us to observe how the bimodal survival distribution occurs for cells that did not grow as filaments, whereas cells that filament increase their survival probability gradually as they have more initial GFP (see also Figure \textcite{ref}(fig:03-temporal-metric-distribution-gfp-1).





\begin{figure}[H]
\includegraphics[width=1\linewidth]{downloadFigs4latex__main/04-gfp-survival-probability-1} \caption[Plasmid initial GFP survival probability.]{\textbf{Plasmid initial GFP survival probability.} We calculated the survival probability after comparing the population distributions of GFP with those of the cells that managed to survive. To assess survival by GFP, we only used plasmid cells. For non-filamented cells (blue dots), a bell forms with an upturned tail. On the other hand, for the filamented cells (red dots), a continuous increase in survival is shown just when it seems that the probability of the non-filamented cells has a decrease. In global, much GFP has higher resistance, but an average GFP value without filamentation also increases the probability of survival.}\label{fig:04-gfp-survival-probability-1}
\end{figure}

As in Figure \ref{fig:04-gfp-survival-probability-1}, in Figure \ref{fig:04-length-survival-probability-1}, we show the survival probability given an initial length.
We observe that survival is higher for cells that did not grow as filament if the initial length was less than the average.
In contrast, for filamented cells, the probability of survival increased as cells length was longer at the beginning of the experiment (see also Figure \ref{fig:03-temporal-metric-distribution-length-1}).
However, it is noteworthy that the probability of survival had a limit in which a higher initial length meant a lower probability of survival (see red dotted lines in Figure \ref{fig:04-length-survival-probability-1}).





\begin{figure}[H]
\includegraphics[width=1\linewidth]{downloadFigs4latex__main/04-length-survival-probability-1} \caption[Plasmid initial length survival probability.]{\textbf{Plasmid initial length survival probability.} We calculated the survival probability after comparing the population distributions of length with those of the cells that managed to survive. For non-filamented cells (blue dots), the survival probability is higher for those cells with initial lengths and small. It seems to decrease with a more extensive initial size. For their part, for filamented cells (red dots), the probability of survival increases according to their length but then declines when the cells are too long at first (see red dotted line). Therefore, in general, a small and moderate length or an initial length already filamented from the beginning increases the chances of survival.}\label{fig:04-length-survival-probability-1}
\end{figure}

\hypertarget{discussion}{%
\section{Discussion}\label{discussion}}

Here, we evaluated different variables that could determine cell survival upon exposure to toxic agents by studying two experimental populations of \emph{E. coli}, one strain with a resistance gene on the chromosome and the other on multicopy plasmids.
We identified two variables that are predominantly responsible for cell survival: cell length and GFP amount related to the cell's inherent resistance to the toxic agent and heterogeneity in response times.

On the other hand, as other studies have already mentioned {[}x{]}, we examined cell activity and youth in a minimalistic way.
While the distribution of the number of divisions exemplifies a broader and more uniform range for the surviving cells, the cells that died showed a tendency to a lower number of divisions.
However, for the study of cellular youth at the time of exposure to the toxic agent, the results did not show a clear pattern of behavior for cell fate determination.
Therefore, it would be interesting to study cellular youth at a higher level of complexity in future studies to understand its contribution to cell survival.

Interestingly, when we used temporal measurements of cell length, GFP, DS-Red, and if a cell divided, we could recapitulate, for the most part, the fates of cellular states (see Sections \ref{cell-length-and-the-amount-of-gfp-are-crucial-in-determining-cell-survival} and \ref{increasing-the-complexity-of-the-system-and-analyzing-it-in-an-unsupervised-way-allows-a-correct-classification-of-cell-states}).
Thus, increasing the system's complexity led to better clustering cell states, but not how these factors interact biologically in determining cell survival.
Therefore, we decided to postulate a mathematical model that helps us understand the critical components in cell survival.

\hypertarget{filamentation-abstraction}{%
\chapter{Models to the rescue; filamentation abstraction}\label{filamentation-abstraction}}

\minitoc 

\noindent Scientists have extensively studied the mechanisms that orchestrate the growth and division of bacterial cells. Cells adapt their shape and dimensions in response to variations in the intracellular and extracellular environments by integrating information about the presence of nutrients or harmful agents in the decision to grow or divide. Filamentation is a process that occurs when rod-shaped cells stop dividing but continue to grow, thus producing elongated cells. Some cells can naturally grow as filamentous, while others only do so under stressful conditions. Here we use mathematical modeling and computational simulations to evaluate a toxic agent's intracellular concentration as a function of cell length. We show that filamentation can act as a strategy that promotes the resilience of a bacterial population under stressful environmental conditions.

\hypertarget{introduction-2}{%
\section{Introduction}\label{introduction-2}}

By integrating information from the environment, cells can alter their cell cycle. For instance, some cells arrest the cell division in the presence of toxic agents but continue to grow. Previous studies have shown that this filamentation phenomenon provides a mechanism that enables cells to cope with stress, which leads to an increase in the probability of survival \autocite{justiceMorphologicalPlasticityBacterial2008}. For example, filamentation can be a process capable of subverting innate defenses during urinary tract infection, facilitating the transition of additional rounds of intracellular bacterial community formation \autocite{justiceFilamentationEscherichiaColi2006}.

Although filament growth can help mitigate environmental stress (e.g., by activating the SOS response system \autocite{justiceMorphologicalPlasticityBacterial2008}), the evolutionary benefits of producing elongated cells that do not divide are unclear. Here, we proposed a mathematical model based on ordinary differential equations that explicitly considers the concentration of intracellular toxin as a function of the cell's length (see Equation \eqref{eq:model-equation}). The model is built based on the growth ratio of measurements of the surface area (\(SA\)) and the cell volume (\(V\)), whereby the uptake rate of the toxin depends on the \(SA\). However, \(V's\) rate of change for \(SA\) is higher than \(SA\) for \(V\), which results in a transient reduction in the intracellular toxin concentration (see Figure \ref{fig:cell-dimensions-relationship}). Therefore, we hypothesized that this geometric interpretation of filamentation represents a biophysical defense line to increase the probability of a bacterial population's survival in response to stressful environments.





\begin{figure}[H]
\includegraphics[width=1\linewidth]{downloadFigs4latex__main/cell-dimensions-relationship} \caption[Cell dimensions relationship.]{\textbf{Cell dimensions relationship.} We evaluated the resulting geometric properties on a grid of side lengths and radii with a pill-shaped cell. We can see that by maintaining a constant radius (typical case in bacteria such as \emph{E. coli}) and increasing the side length, the surface area / volume relationship (\(SA/V\)) tends to decline since the \(V\) will grow at a higher rate than the \(SA\).}\label{fig:cell-dimensions-relationship}
\end{figure}

\hypertarget{filamentation-model}{%
\section{Filamentation model}\label{filamentation-model}}

Let us assume the shape of cells is a cylinder with hemispherical ends. Based on this geometric structure, a nonlinear system of differential equations governing filamentation can be written as follows:

\begin{equation}
\begin{split}
\frac{dT_{int}}{dt} &= T_{sa} \cdot (T_{ext}(t) - T_{vol}) - \alpha \cdot T_{ant} \cdot T_{int} \\
\frac{dL}{dt} &= 
  \begin{cases} 
    \beta \cdot L,& \text{if } T_{int} \geq T_{sos},  t \geq \tau_{sos} + \tau_{delay} \text{ and } L < L_{max}  \\
    0,            & \text{otherwise}
  \end{cases}
\end{split}
\label{eq:model-equation}
\end{equation}

It considers the internal toxin (\(T_{int}\)) and the cell length (\(L\)) as variables. \(T_{sa}\) and \(T_{vol}\) represent the surface area and volume of the toxin in the cell, respectively. \(T_{ext}(t)\) is a function that returns the amount of toxin in the cell medium. \(T_{anti}\) and \(\alpha\) symbolize the amount of antitoxin and its efficiency rate, respectively. \(\beta\) as the rate of filamentation. \(L_{max}\) is the maximum size that the cell can reach when filamentation is on. \(T_{sos}\) and \(T_{kill}\) are thresholds for filamentation and death, respectively. Finally, \(\tau_{delay}\) is the amount of time required to activate filamentation after reaching the \(T_{sos}\) threshold.

\hypertarget{results-1}{%
\section{Results}\label{results-1}}

\hypertarget{filamentation-provides-transient-resistance-under-stressful-conditions}{%
\subsection{Filamentation provides transient resistance under stressful conditions}\label{filamentation-provides-transient-resistance-under-stressful-conditions}}

When growing rod-shaped bacterial cells under constant conditions, the distribution of lengths and radii is narrow \autocite{schaechterGrowthCellNuclear1962}. However, when growing under stress conditions, some cells produce filaments \autocite{schaechterDependencyMediumTemperature1958}. This phenomenon may depend on the SOS response system \autocite{bosEmergenceAntibioticResistance2015}, which can repair DNA damage, giving the cell greater chances of recovering and surviving under stress conditions. Besides, the SOS response has been reported to have precise temporal coordination in individual bacteria \autocite{friedmanPreciseTemporalModulation2005}. Among the stress conditions that can trigger the SOS response is exposure to beta-lactam antibiotics \autocite{millerSOSResponseInduction2004}.

In general, filamentation has been studied as an unavoidable consequence of stress. However, we considered filamentation an active process that produces the first line of defense against toxic agents. Therefore, a differential equation model was proposed that assesses the change in the amount of internal toxin as a function of cell length. At the core of this model, we include the intrinsic relationship between the surface area and the capsule volume since it is vital in determining cell size \autocite{harrisRelativeRatesSurface2016}.

In figure \ref{fig:filamentation-model-ramp-signal}, cells grow in a ramp-shaped external toxin signal without considering a toxin-antitoxin system. As time progresses, the toxin in the external environment increases, so the cell raises its internal toxin levels. At approximately time \(22\), any cell reaches \(T_{sos}\). The control cell (unable to filament) and the average cell (capable of filamenting) reach the death threshold, \(T_{kill}\), at times \(31\) and \(93\) (hatched and solid black lines), respectively. Therefore, under this example, the cell has increased its life span three times more than in control by growing as a filament (green shaded area versus orange shaded area). In turn, figure \ref{fig:filamentation-model-ramp-signal} also shows stochastic simulations of the system in the intake of internal toxins. Considering that cell growth and death processes are inherently stochastic, stochastic simulations would be a better approximation. However, from now on, we will continue with the study of the deterministic model.





\begin{figure}[H]
\includegraphics[width=1\linewidth]{downloadFigs4latex__main/filamentation-model-ramp-signal} \caption[Effect of filamentation on intracellular toxin concentration.]{\textbf{Effect of filamentation on intracellular toxin concentration.} In the presence of an extracellular toxic agent, the intracellular concentration of the toxin (\(T_{int}\)) increases until reaching a damage threshold that triggers filamentation (\(T_{sos}\), blue point), increasing cell length (\(L\)). When filamentation is on, the cell decreases \(T_{int}\) due to the intrinsic relationship between surface area and cell volume. When the cell reaches its maximum length, it eventually dies if the stressful stimulus is not removed (\(T_{kill}\), red dot). The hatched line represents a cell that can not grow as filament. The orange shaded area is the time between stress and the non-filament cell's death, while the green shaded area represents the temporal gain for doing so. The blue background lines represent stochastic simulations of the same system.}\label{fig:filamentation-model-ramp-signal}
\end{figure}

\hypertarget{filamentation-increases-the-minimum-inhibitory-concentration}{%
\subsection{Filamentation increases the minimum inhibitory concentration}\label{filamentation-increases-the-minimum-inhibitory-concentration}}

Antimicrobial resistance (AMR) can be considered one of the most critical health problems of the century. That is, microorganisms' ability to grow despite exposure to substances designed to inhibit their growth or kill them. In April 2014, the World Health Organization (WHO) published its first global report on AMR surveillance \autocite{EditorialBoard2014}. Taking out of the darkness a common fear, a possible post-antibiotic future in which common infections or minor injuries can kill. Therefore, understanding the mechanisms of avoiding antibiotic action is essential for producing knowledge and developing strategies that reduce the generation of resistant bacteria.

A classic experiment in laboratories finds the concentration that inhibits bacterial growth through exposure to different toxin doses. The concentration found is known as the minimum inhibitory concentration (MIC) \autocite{andrewsDeterminationMinimumInhibitory2002}. Bacteria are capable of modifying their MIC through various mechanisms, for example, mutations \autocite{lambertBacterialResistanceAntibiotics2005}, impaired membrane permeability \autocite{satoOuterMembranePermeability1991}, flux pumps \autocite{webberImportanceEffluxPumps2003}, toxin-inactivating enzymes \autocite{wrightBacterialResistanceAntibiotics2005}, and plasticity phenotypic \autocite{justiceMorphologicalPlasticityBacterial2008}. The latter is our phenomenon of interest because it considers the change in shape and size, allowing us to study it as a strategy to promote bacterial survival.

We decided to analyze the MIC change caused by filamentation through stable exposure experiments of different toxin amounts at other exposure times. Computational simulations show that when comparing cells unable to filament with those that can, there is an increase in the capacity to tolerate more generous amounts of toxin, increasing their MIC (see Figure \ref{fig:increased-time-resistance}). Therefore, it confers a gradual increase in resistance beyond filamentation's role in improving the cell's life span as the exposure time is longer.





\begin{figure}[H]
\includegraphics[width=1\linewidth]{downloadFigs4latex__main/increased-time-resistance} \caption[Effect of filamentation on minimum inhibitory concentration (MIC).]{\textbf{Effect of filamentation on minimum inhibitory concentration (MIC).} By exposing a cell to different toxin concentrations with stable signals, the cell achieves a set MIC for conditions without or with filamentation (separation between stressed and dead state) for each exposure time, without representing a change for the normal state cells' points (blue zone). Thus, the green line represents a gradual MIC increase when comparing each MIC between conditions for each exposure time.}\label{fig:increased-time-resistance}
\end{figure}

\hypertarget{heterogeneity-in-the-toxin-antitoxin-system-represents-a-double-edged-sword-in-survival-probability}{%
\subsection{Heterogeneity in the toxin-antitoxin system represents a double-edged sword in survival probability}\label{heterogeneity-in-the-toxin-antitoxin-system-represents-a-double-edged-sword-in-survival-probability}}

One of the SOS response system properties is that it presents synchronous activation times within homogeneous populations \autocite{friedmanPreciseTemporalModulation2005}. It has constant gene expression rates that help it cope with stress; however, it is possible to introduce variability by considering having multicopy resistance plasmids \autocite{million-weaverMechanismsPlasmidSegregation2014}. Therefore, the response times would have an asynchronous behavior at the global level but synchronous at the local level.

To include this observation into the model, we include a negative term to the internal toxin representing a toxin-antitoxin system. Therefore, the model now has an efficiency rate of the antitoxin and a stable amount per cell. We simulate the effect of the toxin-antitoxin system variation within a \(1000\) cell population; we initialize each one with similar initial conditions, except for the amount of internal antitoxin, defined as \(T_{anti} \sim N(\mu, \sigma)\). Considering that \(T_{anti}\) values \(< 0\) are equal to \(0\). For each experiment,\(\mu = 25\), while it was evaluated in the range \([0-20]\). For the generation of pseudo-random numbers and to ensure the results' reproducibility, the number \(42\) was considered seed.

As shown in Figure \ref{fig:variability-toxin-antitoxin}, when we compare heterogeneous populations in their toxin-antitoxin system, we can achieve different population dynamics, that is, changes in the final proportions of cell states; normal, stressed, and dead. This difference is because the cell sometimes has more or less antitoxin to handle the incoming stress situation.





\begin{figure}[H]
\includegraphics[width=1\linewidth]{downloadFigs4latex__main/variability-toxin-antitoxin} \caption[Variability in the toxin-antitoxin system produces different proportions of cell states.]{\textbf{Variability in the toxin-antitoxin system produces different proportions of cell states.} Histograms represent the distribution of antitoxin quantity, while the curves represent the population's fraction over time. The cell will start to filament after reaching a certain internal toxin threshold, \(T_{sos}\). Therefore, the expected global effect on the population's response times based on the amount of antitoxin is asynchronous, while at the local level, it is synchronous. Consequently, different proportions are presented in the cellular states since some cells will activate the filamentation system before and others later.}\label{fig:variability-toxin-antitoxin}
\end{figure}

Considering that the toxin-antitoxin system's variability can modify the proportions of final cell states, a question arose about heterogeneity levels' global behavior. To answer this question, we evaluate the probability of survival for each population, defined by its distribution of antitoxin levels. In this way, we characterized the population survival probability function into three essential points according to its effect: negative, invariant, and positive (see Figure \ref{fig:survival-probability}). Furthermore, these points are relative to the homogeneous population's death time in question (\(\tau_{kill}\)): when \(t < \tau_{kill}\) will represent a detrimental effect on survival, \(t = \tau_{kill}\) will be independent of variability, and \(t > \tau_{kill}\) will be a beneficial point for survival. Therefore, we concluded that the effect of heterogeneity in a toxin-antitoxin system represents a double-edged sword.





\begin{figure}[H]
\includegraphics[width=1\linewidth]{downloadFigs4latex__main/survival-probability} \caption[Effect of variability on the toxin-antitoxin system.]{\textbf{Effect of variability on the toxin-antitoxin system.} The color of the heatmap is representative of the fraction of living cells at exposure time. The white vertical line represents the death time of the homogeneous population (\(\tau_{kill}\)). At \(t < \tau_{kill}\), it is shown that the fraction of survivors decreases as the variability in the population increases. When \(t = \tau_{kill}\), the variability does not affect the fraction of survivors, but it represents a percentage improvement for the homogeneous population. Finally, when \(t > \tau_{kill}\), the heterogeneity of the population favors survival.}\label{fig:survival-probability}
\end{figure}

\hypertarget{discussion-1}{%
\section{Discussion}\label{discussion-1}}

Today, there have been advancements in the number of techniques that have allowed it to extend quantitative analyses to individual cells' dynamic observations \autocite{camposConstantSizeExtension2014,meldrumFacultyOpinionsRecommendation2005,sliusarenkoHighthroughputSubpixelPrecision2011,taheri-araghiCellSizeControlHomeostasis2017,ursellRapidPreciseQuantification2017}. Therefore, studying their cellular behavior every day from medium to medium can be somewhat reproducible, facilitating the association of complex biological functions in simple mathematical equations \autocite{neidhardtBacterialGrowthConstant1999}.

Here, we proposed a mathematical model showing that filamentation could serve as a population's resilience mechanism to stress conditions. Finding that filamentation's net effect generates an additional window of time for the cell to survive, decreasing the toxin's intracellular concentration. However, we also found that a side effect of filamentation is to increase the cell's minimum inhibitory concentration. On the other hand, when we introduce variability in the amount of antitoxin in a cell population, we found that heterogeneity can be a double-edged sword, sometimes detrimental and sometimes beneficial, depending on the time of exposure to the toxic agent.

Due to the above, despite being simple, the model could have the ability to recapitulate the behavior seen in nature from variables that we can calculate easily with single-cell measurements. However, in other situations, it could be helpful to consider the addition of variables such as cell wall production and peptidoglycans' accumulation, among others. Notwithstanding the lack of parameters that are a little closer to reality, confirming that the model can work under experimental conditions would represent an achievement due to its explanatory simplicity, starting, in this way, the path of the study of filamentation oriented to the ecology of stress.

\hypertarget{chapter-discussion}{%
\chapter{Discussion}\label{chapter-discussion}}

\minitoc 

\noindent

\startappendices

\hypertarget{the-first-appendix}{%
\chapter{The First Appendix}\label{the-first-appendix}}


%%%%% REFERENCES
\setlength{\baselineskip}{0pt} % JEM: Single-space References

{\renewcommand*\MakeUppercase[1]{#1}%
\printbibliography[heading=bibintoc,title={\bibtitle}]}


\end{document}
